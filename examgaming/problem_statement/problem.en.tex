\problemname{Exam Gaming}
Rasmus, Magnus, and Jonas have an upcoming exam in Algorithmic Problem Solving with their professor Martin. For their exam, they have been given a list of Kattis\footnote{An online programming problem solving platform with a big community and numerous problems, available at open.kattis.com} problems and have to solve a number of them. They all consider themselves Kattis experts and hence are not concerned about their ability to solve all the problems. Their arrogance has, however, led them to spend their time doing anything but working on their exam.

With only 100 hours before deadline, they suddenly realise that their exam won't write itself, and it is now clear to them how unlikely it is that they will be able to complete all the problems in time. Yet, being the industrious students they obviously are, they have come up with a plan. They hope that Martin might forgive them for not doing all problems, if the ones they complete are solved in a spectacularly impressive way.

Being self-proclaimed Kattis experts also means that the group members all have very strong opinions on which approach they should use to solve each problem. In some cases they can come to agreement, but in other cases each of the three group members might propose at most one approach. For example, they could quickly implement a brute-force solution, but a dynamic programming solution, that takes a longer time to implement, would make Martin happier. They do realise that it will not make Martin any happier, if they solve the same Kattis problem twice or thrice using different approaches. 

To figure out which problems to solve, and with what approach, the group has created a ranking of all problems, approaches, happy-points (a measure of how happy Martin will be), and the time it takes to implement.

The group is very tight-knit and can only work on one problem at a time, all together. Help them figure out how many happy-points they can achieve before the deadline.

\section*{Input}
The first line of input contains an integer $n$, $1\leq n\leq 100000$.
The following $n$ lines each contain a unique problem/approach pair, with the values $K$, $M$, $P$, and $T$. 
\begin{itemize}
    \item $K$: Kattis problem name. A single word using lowercase characters a–z, no longer than 100 characters.
    \item $M$: Proposed approach. A single word using lowercase characters a–z, no longer than 100 characters.
    \item $P$: An integer $1\leq P\leq 200$. The amount of happy-points scored for solving the problem with this approach.
    \item $T$: An integer $1\leq T\leq 100$. The number of hours it takes to implement a solution to this problem with this approach.
\end{itemize}
No problem will have more than 3 approaches to solving.

\section*{Output}
Your program should find the best possible combination of problems such that the sum of all points $P$ is the highest possible value, while staying within the time limit.
Output a single integer: The greatest number of happy-points the group can archive within the time limit.
